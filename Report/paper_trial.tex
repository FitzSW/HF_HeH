% ****** Start of file aipsamp.tex ******
%
%   This file is part of the AIP files in the AIP distribution for REVTeX 4.
%   Version 4.1 of REVTeX, October 2009
%
%   Copyright (c) 2009 American Institute of Physics.
%
%   See the AIP README file for restrictions and more information.
%
% TeX'ing this file requires that you have AMS-LaTeX 2.0 installed
% as well as the rest of the prerequisites for REVTeX 4.1
% 
% It also requires running BibTeX. The commands are as follows:
%
%  1)  latex  aipsamp
%  2)  bibtex aipsamp
%  3)  latex  aipsamp
%  4)  latex  aipsamp
%
% Use this file as a source of example code for your aip document.
% Use the file aiptemplate.tex as a template for your document.
\documentclass[%
 aip,
% jmp,
% bmf,
% sd,
% rsi,
 amsmath,amssymb,
%preprint,%
 reprint,%
%author-year,%
%author-numerical,%
% Conference Proceedings
]{revtex4-1}

\newcommand{\p}{\partial}
\newcommand{\lbar}{\left|}
\newcommand{\rbar}{\right|}
\newcommand{\lbr}{\left<}
\newcommand{\rbr}{\right>}
% \usepackage{hyperref}
\usepackage{fontspec}
\newfontfamily\code{Courier New}
\usepackage{graphicx}% Include figure files
\usepackage{dcolumn}% Align table columns on decimal point
\usepackage{bm}% bold math
%\usepackage[mathlines]{lineno}% Enable numbering of text and display math
%\linenumbers\relax % Commence numbering lines

\usepackage[utf8]{inputenc}
\usepackage[T1]{fontenc}
\usepackage{mathptmx}
\usepackage{etoolbox}
% \usepackage{cite}

%% Apr 2021: AIP requests that the corresponding 
%% email to be moved after the affiliations
\makeatletter
\def\@email#1#2{%
 \endgroup
 \patchcmd{\titleblock@produce}
  {\frontmatter@RRAPformat}
  {\frontmatter@RRAPformat{\produce@RRAP{*#1\href{mailto:#2}{#2}}}\frontmatter@RRAPformat}
  {}{}
}%
\makeatother
\begin{document}

\preprint{AIP/123-QED}

\title[MENG 25510 Final]{MENG 25510 Final Report:\\ Hartree-Fock on
HeH$^{+}$ with 6-311G}
% Force line breaks with \\
\author{N.\ R.\ Dohrmann and S.\ W.\ Fitz}
 % \altaffiliation{University of Chicago, The College}%Lines break automatically or can be forced with \\
 % \email{ndohrmann@uchicago.edu}


\date{30 May 2022}% It is always \today, today,
             %  but any date may be explicitly specified

\begin{abstract}
Please see the corresponding code at \url{https://github.com/FitzSW/HF_HeH}
for this project.
\end{abstract}

\maketitle

\section{Introduction}
This is a reference document to go with our submission for the MENG 25510
final coding project, for which we chose to implement the Hartree-Fock (HF)
Self-Consistent Field (SCF) method on a HeH$^{+}$ molecular system using the
6-311G basis set on both atoms. The geometry at which we performed the
calculation was optimized at the CCSD/aug-cc-pZTZ level of theory using
Gaussian 16\cite{g16}. The result of our code is then compared to an energy
found with
PySCF\cite{sun2020recent,https://doi.org/10.1002/wcms.1340,https://doi.org/10.1002/jcc.23981}
using the same geometry and basis set.

\section{Hartree-Fock Algorithm and Code Details}
We closely follow the suggested implementation scheme given in Szabo and
Ostlund\cite{szabo2012modern}, in which explicit formulas and matrix
algorithms are given. Any of the following formulas can be found in this
reference text if not explicitly stated otherwise.

\subsection{Program Control Flow} 
First, the input geometry and basis set files are read into the main program
({\code main.f90}), which are then used to construct a 1D array of the
``orbitals'' derived type (written in {\code orbitals.f90}), which each include
the name of its host atom, the orbital angular momentum type (e.g. ``s''), the
coordinates of the host atom, the length of the contraction, as well as arrays
of coefficients and exponential factors for the contraction.

From here, we then calculate the stored integrals to populate the matrices 
$\bold{S}$, $\bold{T}$, $\left\{\bold{V}^{\text{nuc.}}\right\}$, and $\bold{TE}$,
with the last representing the two-electron integrals. The eigenvalue problem
that we are trying to solve is recommended to be handled via a transformation
to a set of orthogonalized orbitals via a matrix $\bold{X}$ obtained from 
the \emph{symmetric orthogonalization} scheme
%
\begin{equation}
\bold{X} = \bold{S}^{-1/2}
\end{equation}
%
The necessary matrix computations are done with assistance from the C++
numerics library Eigen\cite{eigenweb}, which is also later used to find the 
eigenvalues and eigenvectors of matrices at each cycle of the SCF procedure
when solving 
\begin{equation}
\bold{F}'\bold{C}' = \bold{C'}\bold{\epsilon}
\end{equation}
with $\bold{F}'$ as the transformed Fock matrix, 
$\bold{C}'$ as the transformed coefficient matrix, and $\bold{\epsilon}$ as a
diagonal matrix of orbital energies.

After the transformed coefficient matrix is found, it is reverted back to the
original basis via $\bold{C} = \bold{X}\bold{C}'$. A real attempt at the density
matrix $\bold{P}$ can now be found, as well as $\bold{G}$ for the next cycle
of the SCF procedure using a new Fock matrix.  This cycle is repeated
iteratively until the largest change in any element of the density matrix is
less than $10^{-4}$.  Finally, once we have obtained converged matrices, a
numerical value for the ground state electronic energy of the occupied
molecular orbital is
\begin{equation}
E_0^{\text{elec.}} = \frac{1}{2} \sum_{\mu\nu} \bold{P}_{\nu\mu} \left(
\bold{H}_{\mu\nu}^{\text{core}} + \bold{F}_{\mu\nu} \right)
\end{equation}
which can then be added to the nuclear repulsion energy 
$E_0 = E_0^{\text{elec.}} + E_0^{\text{nuc.}}$.


\section{Use and Installation}
In order to run our code, please clone the git repository that is linked in
the abstract of this document. These instructions will assume that you are
 on a *nix system (i.e.\ you have access to {\code wc} and {\code rm} as
 system commands) and have {\code gfortran} and a C++ compiler such as 
 {\code clang} that has access to the Eigen header files. If you don't
have the header files,  then run either of the below commands depending if you
are on a Mac or are on Linux (Midway preferred)
$$ \text{{\code ./mac\_eigen.sh} \hspace{0.5cm}{\code ./midway\_eigen.sh}}$$
These will install the necessary files via git.  \textbf{Please make sure to
do this \emph{before} running the build script.} Then, inside the cloned
directory, run one of the following commands: 
%
$$ \text{{\code ./mac\_build.sh}} \hspace{0.5cm}
 \text{{\code ./midway\_build.sh}} $$
to build and execute the project. The results of the SCF procedure will be
 written to file in {\code hf\_out.out}.

\section{Results and Discussion}
We found a final electronic energy of () with a nuclear repulsion energy of
() for a total molecular energy of () [Ha.]. Please see a comparison of our
results to the benchmark and PySCF values in Table \ref{tab:hf_results}.

\begin{table}[]
\caption{HF Results Comparison on HeH$^{+}$ by Method}
\label{tab:hf_results}
\begin{ruledtabular}
\begin{tabular}{|c|c|c|}
\hline
\multicolumn{1}{|l|}{\textbf{Method / Data Source}} &
\multicolumn{1}{c|}{\textbf{Bond Length\footnote{Optimzed with
CCSD/aug-cc-pVTZ, this value used with both HF/6-311G calculations.} [Å] \ \ \ \ }} 
& \multicolumn{1}{l|}{\textbf{Total $E_0$, [Ha.]}} \\ \hline
CCSD/aug-cc-pVTZ\footnote{Gaussian 16}      & .776140  & -2.9753932    \\ \hline 
HF/6-311G\footnote{PySCF}     & .776140     & -2.9164905    \\ \hline
HF/6-311G\footnote{This Report}  & .776140  &          \\ \hline
\end{tabular}
\end{ruledtabular}
\end{table}


\section{Conclusion}
If this project were to be extended for more general use, the basis sets and
geometries of additional systems could easily be read into the main program
in the current state of the code. The main area for improvement that would
have to be more thoroughly adapted are the evaluation of molecular integrals 
involving atomic orbitals of angular momentum $l > 0$, which are much more
complex in their evaluation than the $s$ type orbitals that are considered in
this report. Additionally, there is further room for optimization in the
linear algebra functionality of this project. Currently, matrices must be
written to file, processed, and read back into the main executable at the
start of the program and at each stage of the SCF cycle. Clearly, this is not
the most efficient way to handle matrix computations, and an improved code
would ideally include LAPACK functions and subroutines to be used within
Fortran itself.


% \nocite{*}
\bibliography{sources}

\end{document}
