% ****** Start of file aipsamp.tex ******
%
%   This file is part of the AIP files in the AIP distribution for REVTeX 4.
%   Version 4.1 of REVTeX, October 2009
%
%   Copyright (c) 2009 American Institute of Physics.
%
%   See the AIP README file for restrictions and more information.
%
% TeX'ing this file requires that you have AMS-LaTeX 2.0 installed
% as well as the rest of the prerequisites for REVTeX 4.1
% 
% It also requires running BibTeX. The commands are as follows:
%
%  1)  latex  aipsamp
%  2)  bibtex aipsamp
%  3)  latex  aipsamp
%  4)  latex  aipsamp
%
% Use this file as a source of example code for your aip document.
% Use the file aiptemplate.tex as a template for your document.
\documentclass[%
 aip,
% jmp,
% bmf,
% sd,
% rsi,
 amsmath,amssymb,
%preprint,%
 reprint,%
%author-year,%
%author-numerical,%
% Conference Proceedings
]{revtex4-1}

\newcommand{\p}{\partial}
\newcommand{\lbar}{\left|}
\newcommand{\rbar}{\right|}
\newcommand{\lbr}{\left<}
\newcommand{\rbr}{\right>}
\usepackage{graphicx}% Include figure files
\usepackage{dcolumn}% Align table columns on decimal point
\usepackage{bm}% bold math
%\usepackage[mathlines]{lineno}% Enable numbering of text and display math
%\linenumbers\relax % Commence numbering lines

\usepackage[utf8]{inputenc}
\usepackage[T1]{fontenc}
\usepackage{mathptmx}
\usepackage{etoolbox}
% \usepackage{cite}

%% Apr 2021: AIP requests that the corresponding 
%% email to be moved after the affiliations
\makeatletter
\def\@email#1#2{%
 \endgroup
 \patchcmd{\titleblock@produce}
  {\frontmatter@RRAPformat}
  {\frontmatter@RRAPformat{\produce@RRAP{*#1\href{mailto:#2}{#2}}}\frontmatter@RRAPformat}
  {}{}
}%
\makeatother
\begin{document}

\preprint{AIP/123-QED}

\title[MENG 25510 Final]{MENG 25510 Final Report:\\ Hartree-Fock on
HeH$^{+}$ with 6-311G}
% Force line breaks with \\
\author{Noah Dohrmann and Sullivan Fitz}
 % \altaffiliation{University of Chicago, The College}%Lines break automatically or can be forced with \\
 % \email{ndohrmann@uchicago.edu}


\date{30 May 2022}% It is always \today, today,
             %  but any date may be explicitly specified

\begin{abstract}
Please see the corresponding code at \url{https://github.com/FitzSW/HF_HeH}
for this project.
\end{abstract}

\maketitle

\section{Introduction}
This is a reference document to go with our submission for the MENG 25510
final coding project, for which we chose to implement the Hartree-Fock (HF)
Self-Consistent Field (SCF) method on a HeH$^{+}$ molecular system using the
6-311G basis set on both atoms. The geometry at which we performed the
calculation was optimized at the CCSD/aug-cc-pZTZ level of theory using
Gaussian 16\cite{g16}. The result of our code is then compared to an energy
found with
PySCF\cite{sun2020recent,https://doi.org/10.1002/wcms.1340,https://doi.org/10.1002/jcc.23981}
using the same geometry and basis set.

\section{Hartree-Fock Algorithm and Code Details}
We closely follow the suggested implementation scheme given in Szabo and
Ostlund\cite{szabo2012modern}, in which explicit formulas and matrix
algorithms are given.  

\subsection{Program Control Flow} 
First, the input geometry and basis set files are read into the main program
(``main.f90''), which are then used to construct a 1D array of the
``orbitals'' derived type (written in ``orbitals.f90''), which each include
the name of its host atom, the orbital angular momentum type (e.g. ``s''), the
coordinates of the host atom, the length of the contraction, as well as arrays
of coefficients and exponential factors for the contraction.

From here, we then calculate the stored integrals to populate the matrices 
$\bold{S}$, $\bold{T}$, $\left\{\bold{V}^{\text{nuc.}}\right\}$, and $\bold{TE}$,
with the last representing the two-electron integrals. The eigenvalue problem
that we are trying to solve is recommended to be handled via a transformation
to a set of orthogonalized orbitals via a matrix $\bold{X}$ obtained from 
the \emph{symmetric orthogonalization} scheme
%
\begin{equation}
\bold{X} = \bold{S}^{1/2}
\end{equation}
%
The necessary matrix computations are done with assistance from the C++
numerics library Eigen\cite{eigenweb}, which is also later used to find the 
eigenvalues and eigenvectors of matrices at each cycle of the SCF procedure.


\section{Use and Installation}
In order to run our code, please clone the git repository that is linked in
the abstract of this document. Then, run the command: 


% \nocite{*}
\bibliography{sources}

\end{document}
