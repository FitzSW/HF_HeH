% ****** Start of file aipsamp.tex ******
%
%   This file is part of the AIP files in the AIP distribution for REVTeX 4.
%   Version 4.1 of REVTeX, October 2009
%
%   Copyright (c) 2009 American Institute of Physics.
%
%   See the AIP README file for restrictions and more information.
%
% TeX'ing this file requires that you have AMS-LaTeX 2.0 installed
% as well as the rest of the prerequisites for REVTeX 4.1
% 
% It also requires running BibTeX. The commands are as follows:
%
%  1)  latex  aipsamp
%  2)  bibtex aipsamp
%  3)  latex  aipsamp
%  4)  latex  aipsamp
%
% Use this file as a source of example code for your aip document.
% Use the file aiptemplate.tex as a template for your document.
\documentclass[%
 aip,
% jmp,
% bmf,
% sd,
% rsi,
 amsmath,amssymb,
%preprint,%
 reprint,%
%author-year,%
%author-numerical,%
% Conference Proceedings
]{revtex4-1}

\newcommand{\p}{\partial}
\newcommand{\lbar}{\left|}
\newcommand{\rbar}{\right|}
\newcommand{\lbr}{\left<}
\newcommand{\rbr}{\right>}
\usepackage{graphicx}% Include figure files
\usepackage{dcolumn}% Align table columns on decimal point
\usepackage{bm}% bold math
%\usepackage[mathlines]{lineno}% Enable numbering of text and display math
%\linenumbers\relax % Commence numbering lines

\usepackage[utf8]{inputenc}
\usepackage[T1]{fontenc}
\usepackage{mathptmx}
\usepackage{etoolbox}

%% Apr 2021: AIP requests that the corresponding 
%% email to be moved after the affiliations
\makeatletter
\def\@email#1#2{%
 \endgroup
 \patchcmd{\titleblock@produce}
  {\frontmatter@RRAPformat}
  {\frontmatter@RRAPformat{\produce@RRAP{*#1\href{mailto:#2}{#2}}}\frontmatter@RRAPformat}
  {}{}
}%
\makeatother
\begin{document}

\preprint{AIP/123-QED}

\title[MENG 25510 Final]{MENG 25510 Final Report:\\ Hartree-Fock on
HeH$^{+}$ with 6-311G}
% Force line breaks with \\
\author{Noah Dohrmann and Sullivan Fitz}
 % \altaffiliation{University of Chicago, The College}%Lines break automatically or can be forced with \\
 % \email{ndohrmann@uchicago.edu}


\date{30 May 2022}% It is always \today, today,
             %  but any date may be explicitly specified

\begin{abstract}
Please see the corresponding code at \url{https://github.com/FitzSW/HF_HeH}
for this project.
\end{abstract}

\maketitle

\section{Introduction}
This is a reference document to go with our submission for the MENG 25510
final coding project, for which we chose to implement the Hartree-Fock (HF)
Self-Consistent Field (SCF) method on a HeH$^{+}$ molecular system using the
6-311G basis set on both atoms. The geometry at which we performed the
calculation was optimized at the CCSD/aug-cc-pZTZ level of theory using
Gaussian 16.



% \nocite{*}
% \bibliography{../RSC}

\end{document}
